% -*- Mode:TeX -*-
% LaTeX template for CinC papers                   v 1.1a 22 August 2010
%
% To use this template successfully, you must have downloaded and unpacked:
%       http://www.cinc.org/authors_kit/papers/latex.tar.gz
% or the same package in zip format:
%       http://www.cinc.org/authors_kit/papers/latex.zip
% See the README included in this package for instructions.
%
% If you have questions, comments or suggestions about this file, please
% send me a note!  George Moody (george@mit.edu)
%
\documentclass[twocolumn]{cinc}
\usepackage{graphicx}
%\usepackage{subcaption}
\newcommand{\mapthreed}{\textit{map3d }}
\begin{document}

% Keep the title short enough to fit on a single line if possible.
% Don't end it with a full stop (period).  Don't use ALL CAPS.
\title{Electrocardiographic Comparison of Dobutamine and BRUCE Cardiac \\ Stress Testing With High Resolution Mapping in Experimental Models}

% Both authors and affiliations go in the \author{ ... } block.
% List initials and surnames of authors, no full stops (periods),
%  titles, or degrees.
% Don't use ALL CAPS, and don't use ``and'' before the name of the
%  last author.
% Leave an empty line between authors and affiliations.
% List affiliations, city, [state or province,] country only
%  (no street addresses or postcodes).
% If there are multiple affiliations, use superscript numerals to associate
%  each author with his or her affiliations, as in the example below.

\author { Brian Zenger$^{1,2,3,4}$, Wilson W Good$^{1,2,3}$, Jake Bergquist$^{1,2,3}$ Jess D Tate$^{1,2,3}$, Vikas Sharma$^{4}$, \\Rob S MacLeod$^{1,2,3}$\\
\ \\ % leave an empty line between authors and affiliation
$^1$ Scientific Computing and Imaging Institute, University of Utah, SLC, UT, USA \\
$^2$  Nora Eccles Cardiovascular Research and Training Institute, University of Utah, SLC, UT, USA \\
$^3$ Department of Biomedical Engineering, University of Utah, SLC, UT, USA \\
$^4$ School of Medicine, University of Utah, SLC, UT, USA }

\maketitle
laskjdf;laksdjf;lkj
% LaTeX inserts the ``Abstract'' heading in the proper style and
% sets the text of the abstract in italics as required.
\begin{abstract}

    Clinical tests to detect acute myocardial ischemia induce transient
    cardiac stress by means of exercise or pharmaceutical stimulation and
    measure electrical changes of the heart on the body surface via an
    electrocardiogram (ECG).  Such tests assume that both stress mechanisms
    induce identical--—or at least similar—--forms of ischemia. However,
    [|results of] these tests have been known to contradict each other. To
    improve [ECG|electrocardiographic <Pay attention to how you define ECG
    and use it consistently that way.>] detection of myocardial ischemia,
    we must study how varied stressing agents (pharmacological or paced
    stressors) change [ECG|electrocardiographic] signatures. [To do this
    we|We] simultaneously measured electrical recordings within the
    myocardium, on the epicardial surface, and on the body surface. We then
    induced acute, controlled ischemia and monitored the electrical
    response. To create [|the hemodynamic substrate for] ischemia[|,] we
    applied a constant hydraulic occlusion [of|to] the left anterior
    descending coronary artery. We varied the [heart|ischemic] stress with
    two commonly used clinical protocols, the BRUCE and dobutamine stress
    tests. Each episode lasted 15 minutes with stepwise increase in pacing
    rate or pharmacological infusion rate every 3 minutes. Preliminary
    qualitative results suggest significant differences in the recorded
    electrical signal between pacing and pharmacological stress
    mechanisms. Differences include the location and volume of ischemia and
    its temporal development throughout an [ischemic event|stress episode
    <try to use the same words for details like ``episode''>]. These
    results, [although preliminary|and the experimental means used to
    obtain them], are a significant breakthrough in the field with
    simultaneous[|, high density] electrical recordings [on|within] the
    [three major regions on the heart/torso|myocardium and on the heart and
    torso surfaces].

\end{abstract}
% LaTeX inserts the extra space here automatically.

\section{Introduction}
% Section numbering is automatic.  The examples on the next page
% illustrate how to make subsections.

% [|<Note that the spacing \cite commands is important.
% `globally. \cite{Roth2015}' creates a gap while
% `globally.\cite{Roth2015}' produces none.  I like the latter convention
% but the main thing is to be consistent.]

Ischemic heart disease is one of the most common heart pathologies,
effecting over 8 million people globally. \cite{Roth2015} Myocardial
ischemia occurs when the demand for nutrients and perfusion by the heart
outweighs the available supply. This [|imbalance <avoid ambiguity in
reference pronouns by using them rarely>] recreates a supply-demand
mismatch that can lead to devastating long term consequences including
increased risk for myocardial infarction, cardiac arrhythmia, and sudden
cardiac death.\cite{Roth2015} For decades the electrocardiogram (ECG) has
been the primary acute detection method for myocardial
ischemia. \cite{McCarthy1990} However, current [ECG|electrocardiographic]
methods used to detect myocardial ischemia are mediocre at best, with
reported sensitivity and specificity ranging from 50-72\% and 69-90\%,
respectively. \cite{Akkerhuis2011} This poor [ECG|] performance indicates
that many patients are released from clinical care unaware of their
potentially life-threatening condition while others receive care they do
not need. Improvements in the electrical detection of myocardial ischemia
must be made to ensure patients and physicians can be confident in
diagnosing and treating myocardial ischemia early to prevent potentially
fatal long term consequences.

One possible source of this poor ECG[|-based] performance may originate
from different cardiac stressing mechanisms. Clinical tests induce
transient cardiac stress by means of exercise or pharmaceutical stimulation
and measure electrical changes of the heart on the body surface via an
ECG. Such tests assume that both mechanisms induce identical, or at least
similar, forms of ischemia. However, no definitive experiments have been
reported that assess the electrical [differences|effects] produced during
different stressing mechanisms. [Studies that have reported electrical
findings for both stress types have shown different electrical morphologies
of ischemia.|<This is a little vague and lacks a reference, which it really
needs.] These contradictions indicate that the different cardiac stressing
methods may not produce identical regions [|or types] of ischemia and
substantiates a controlled examination of [the methods used to stress the
heart|these methods].

[To improve [ECG|electrocardiographic] detection of myocardial ischemia, we
must study how varied stressing agents (pharmacological or paced stressors)
change ECG signatures.|<This sentence is a restatement of the last
paragraph and fails to summarize the actual content of this paragraph.  I
would leave it out. The next sentence works much better and is an impactful
statement to have clearly stated.>] To date, no experimental model
[provides|has provided] accurate[ly| and comprehensive] sampling of
electrical signals [|from controlled ischemia] within the myocardial
tissue, on the heart surface, and on the body surface [with control precise
ischemic control|<This phrase shift the end-of-sentence emphasis away from
the really interesting statement to a less interesting one. I buried this
less information within the sentence.>]. A model with all of these
components is necessary to understand how the ischemic regions develop
within the heart, and how they manifest on the body surface. For this
study, our goal was to test the differences in myocardial ischemia
development under [controlled experimental conditions. We tested|] two
[|clinical] cardiac stressing mechanisms, pacing the heart [following
average heart rates of | according to the] BRUCE protocol and continuous
dobutamine infusion[ rates of a standard clinical protocol|].


%
%\begin{figure}[h]
%%\centering
%\includegraphics[width=7.9cm]{graph.png}
%\caption{Put the figure legend here, clearly describing the figure.}
%\label{FIGURA1}
%\end{figure}

%Always leave a line space after a figure legend. Avoid background colors as they can make printed figures hard to read.

\section{Methods}

\subsection{Animal Model} 

Swine and canine animal models were selected for this experimental
preparation[. Both species were chosen|] because of their similar cardiac
anatomy, electrical system, and vascular structure to humans. The animals
of each species were [25-35 | <Time to learn about the three different
dashes in LaTex, the hyphen (-), the range (--), and the parenthetical
expression (---).  Here, we use --.> 25--35] kg in weight and 8 months to
several years of age. The animals were purpose [bread|bred] for [the|] use
in experimental research[. All| and all] studies were approved by the
Institutional Animal Care and Use Committee at the University of Utah and
conformed to the Guide for Care and Use of Laboratory Animals. After 12
hours of fasting[|,] the animals were sedated using an intravenous propofol
bolus of [5-8 mg/Kg | 5--8~mg/Kg] in canines or a mixture of Telazol
(4.4~mg/kg), Ketamine (2.2~mg/kg), and Xylazine (2.2~mg/kg) in swine
[through intravenous access|] and then intubated. Once intubated,
isoflurane gas (1-5\%) was used for anesthesia. At the end of the
experiment animals were euthanized while under general anesthesia, with
intravenous Beuthanasia 1~ml/10~Kg. The heart was then removed for further
evaluation.

\subsection{Surgical Procedure}

Following sedation, a sternotomy was performed to expose the thoracic
cavity. The pericardium was opened and the heart was suspended in a
pericardial cradle. Following exposure, a portion of the left anterior
descending coronary artery (LAD) was dissected and a calibrated hydraulic
occluder (Access Technologies, Skokie, IL, USA) was placed around the
dissected portion [of the LAD coronary|]. An atrial pacing clip was then
placed on the appendage of the right [atria|atrium]. Following placement of
the electrical recording equipment (described below), the pericardium was
sutured closed and the sternum was wired and sutured together. To limit air
within the volume conductor, chest tubes were tunneled into the
mediastinal, pleural, and pericardial cavities and held under constant
vacuum suction. The outer layers of dermis were sutured closed and checked
for potential separations. Standard laboratory markers were measured and
recorded throughout the experiment including blood pH,
[PaCO2|${\rm PaCO_2}$ <Another LaTeX trick>], oxygen saturation,
temperature, and blood pressure.

\subsection{Electrical Recording Equipment}

\subsubsection{Electrode Arrays}

Electrical recording equipment was all custom build at the Nora Eccles
Treadwell Cardiovascular Research and Training Institute (CVRTI). The
electrical signals within the myocardium were measured using transmural
plunge needle arrays with 10 electrodes spaced 1.6 or 1.0 mm apart for left
and right ventricular needles[|,] respectively. For these experiments[|,]
[12-25|12--25] needles were placed in the assumed perfusion bed of the LAD
[coronary|] and concentrated on the anterior surface of the heart. The
epicardial potentials were measured using a 247-electrode sock array with
evenly spaced electrodes stitched into a nylon stocking material. The
distance between sock electrodes was approximately 10~mm. The torso surface
electrodes were in linear strips of 12 electrodes evenly spaced at 3~cm
[apart|]. Each electrode had an 11~mm diameter Ag-AgCl sensor embedded in
an epoxy housing with a 2~mm deep gel cavity. The number of strips applied
to the torso surface varied between [6-10|6--10] [electrode strips|]
([72-120|72--120 total] electrodes) depending on [|the body] surface area
[available|accessible] for each animal.

\subsubsection{Data Acquisition}

The potentials from the sock, needle, and torso surface electrodes were
recorded using a custom acquisition system. This system could record
simultaneously from 1024~channels at 1~kHz sampling rate and 12~bit
resolution. Briefly, the acquisition system consisted of multiplexers,
interface circuitry, and a personal computer (PC) hosting a custom program
written in Labview (National Instruments, Austin, TX, USA) that managed the
hardware and allowed continuous signal acquisition. A bandpass filter with
cutoff frequencies at 0.03 and 500~Hz avoided both DC potentials and
aliasing.  Wilson's central terminal leads were used as the remote
reference for all the unipolar signals recorded from the sock, needle[|s],
and torso surface electrodes. Prior to each experiment, calibration signals
were recorded for each channel[ to be gain adjusted|More detail than
needed>].


\subsection{Ischemia Intervention Protocols}

During [an|each] experiment[|,] several transient ischemic interventions
could be induced. Each of these interventions lasted between [8-15|8--15]
minutes and were followed by a 30-minute rest period. The BRUCE exercise
stress was simulated by increasing paced heart rate a set amount above
resting heart rate every three minutes for fifteen minutes. This increase
in heart rate was predetermined from [average increased heart rates
during|] BRUCE stress protocols reported in the
literature. \cite{Okin1986a} The occlusion percentage was fixed throughout
the [15 minute time <You say 8--15 above so this looks contradictory and is
not needed>|stress] interval. The intervention was terminated with the
presence of [|a sequence of] three or more premature ventricular
contractions [in series|]. [Standard clinical dobutamine testing was used
also tested.|] During the dobutamine stress [test|protocol,] the
[model|animal] was continuously infused at a [set dose|a sequence of doses,
each] for three minutes. Dosages [used|] followed the standard [|clinical]
dobutamine stress testing protocols[.|] \cite{Secknus1997} [This|and the]
intervention again lasted 15 minutes or until a [series|sequence] of three
premature ventricular contractions occurred.

\subsubsection{Image Acquisition and Segmentation}

After each experiment[|,] the intact torso was imaged with a clinical
3-Tesla MRI (Seimens Medical) for gross anatomy and electrode
positions. Following the full torso scan, the heart was excised and scanned
with a 7-Tesla MRI scanner (Bruker BIOSPEC 70/30, Billerica, MA) using FISP
(Fast Imaging with Steady-state Precession) and FLASH (Fast Low Angle Shot)
imaging sequences. To visualize fiber orientation in [the |each] heart, a
diffusion-weighted MRI sequence was also performed [for each excised
heart|]. Capitalizing on the combined advantages of both FISP (consistent
volume boundaries) and FLASH sequences (high internal contrast), we
produced [realistic <What does this word mean in this setting?  What is an
`unrealistic' segmentation?>|] geometric segmentations of cardiac tissue,
blood, and transmural plunge needle geometries using the Seg3D open-source
software package (https://www.sci.utah.edu/software/seg3d).

\subsection{Geometric Registration}

%\subsubsection{Landmark Point Recordings}

At the conclusion of each experiment, the locations of the [linear|] torso
surface electrode strips, preselected sock electrodes, [transmural | and]
plunge needle[s | insertion sites] on the cardiac surface were digitally
recorded using a Microscribe three-dimensional digitizer (Solution
Technologies, Oella, MD, USA). In addition, landmark sites including the
location of the occlusion site, [distribution of|] major epicardial
coronary arteries, and the outline of the myocardial shape were also
captured using the digitizer. Once the locations of the plunge needles were
recorded, they were replaced with plastic spacers.


\subsection{Signal Processing and Data Visualization}

The electrical signals recorded during the study were processed in
``Preprocessing Framework for Electrograms Intermittently Fiducialized from
Experimental Recordings'' (PFEIFER) [|program] an open-source MATLAB-based
signal processing platform designed to process bioelectric signals acquired
from [|cardiac] experiments [that include recording electrodes placed in or
around the heart or on the body surface|]. \cite{Rodenhauser2018} Using
PFEIFER[|,] we were able to calibrate, baseline correct, filter, and mark
specific time instances within cardiac signals for analysis.

Finally[|,] the processed experimental [data was|signal were] mapped to the
identified electrode locations within the heart, on the epicardial surface
of the heart, and on the torso surface. Full experimental model datasets
were then visualized using \mapthreed
(https://www.sci.utah.edu/software/map3d) or SCIRun
(https://www.sci.utah.edu/software/scirun) open-source software
packages[. These software packages and tools allowed |, which provided] for
extensive [spacial|spatial] exploration of [ischemia and hypothesis
development|the results].

\section{Results}

[Our experimental model was| With this experimental preparation, we were]
able to simultaneously record from all three aforementioned regions with
high resolution[| and adequate coverage]. [These recordings were then
accurately localized to the correct location on the animal geometry and
visualized.|] Ischemic control was achieved and four transient episodes of
ischemia were induced for each animal [model|<You make the same mistake as
Brett.  An experimental model is the setup not the individual animal>].

[Our results show a | Figure \ref{fig:myo} shows one result to illustrate
the] difference[|s] in the region of ischemia created during the BRUCE and
Dobutamine stress tests. During peak ischemic stress[|,] the region
identified as ischemic [from|based on elevated] ST\%40 potentials was
larger [|in|following] the dobutamine [experiments|protocol] [compared
to|than in] the BRUCE protocol [experiments|]. [(Figure \ref{fig:myo})|]
[These results | Figure \ref{fig:epitorso} shows how these features also]
propagated to the epicardial and body surface[|s with the expected loss of
localization <We need to say something about this, not sure if these are
the best words.>]. [(Figure \ref{fig:epitorso}).|]


\begin{figure}
	\centering
	\includegraphics[width = .45\textwidth]{../Figures/1.png}
	
	\caption{Regions of ischemia detected within the myocardium.[ Red
          hue represents larger ST\%40 values and Blue hue represents lower
          ST\%40 values.|ST\%40 values are indicated by the color, as
          indicated in the scale bar.  The dashed lines indicate the slice
          levels captured in the adjacent projections.] Row 1: Bruce
          protocol. Row 2: Dobutamine Protocol. [|The superior edge of each
          slice corresponds to the anterior surface in the leftmost
          column. ]}
	\label{fig:myo}
\end{figure}

\begin{figure}
	\centering
	\includegraphics[width = .45\textwidth]{../Figures/2.png}

	\caption{Regions of ischemia detected on the epicardial surface and
          the torso surface. [Red hue represents larger ST\%40 values and
          Blue hue represents lower ST\%40 values.|Again, color indicates
          amplitude of the local ST\%40 values] Row 1: Epicardial and
          Torso measurements from a Bruce Protocol. Row 2: Epicardial and
          Torso Measurements from a dobutamine protocol.  [|Results are
          from the same cases as in Figure~\ref{fig:myo} <Is this true?  It
          should be.>] }
	\label{fig:epitorso}
\end{figure}

\section{Discussion}

In this study, we proposed a novel experimental preparation to [better|]
characterize and understand the electrical signals of myocardial ischemia[|
within the heart and on the epicardial and body surfaces]. In specific, we
tested the hypothesis that different clinical cardiac stress tests cause
different electrical signatures detectable on the body surface. Our results
show that [individual stress mechanisms via|] the BRUCE and dobutamine
stress tests produce different amounts [|and spatial distributions] of
ischemia with comparable heart rates. [The characteristic electrical
signals on the body surface were similar in overall pattern but different
in amplitude.| The largest differences were visible within the myocardium,
as shown in the number and locations of extrema in Figure~\ref{fig:myo}.
Epicardial and torso surface differences were also visible but limited to
amplitudes of shared features. ] The dobutamine stress tests produced body
surface signals with significant amounts of depression, while the BRUCE
protocols produced [a|only] relatively mild depression.

[|<Now you are heading in a new direction so a separate paragraph seems
necessary.>] These findings suggest different [stresses on |means of
stressing] the heart require unique diagnostic criteria to detect and
monitor myocardial ischemia. This finding underscores a recurring theme
that the process of ischemia detection and development is more complicated
than conventional explanations have indicated. [Previous studies similar to
this have shown significant deviations from conventional clinical notions
and metrics of myocardial ischemia.| <This is vague and needs a citation.
I assume you mean something like the following:> Our previous results have
already demonstrated that non-transmural ischemia arises from multiple
regions with the myocardium and shows complex spatio-temporal
progression.\cite{}]

Another important breakthrough in this study [is|was] the simultaneous
recordings from within the heart, on the heart surface, and on the body
surface during a controlled ischemic intervention. To date, [no group has
simultaneously recorded from each of these regions with | we know of no
published results documented ] such high resolution [while inducing
controlled myocardial ischemia | with a similar protocol]. The datasets used
in this study will be ideal test datasets for other methods of detecting
ischemia, including electrocardiographic imaging. The high resolution,
ischemic control, and simultaneous recordings in multiple region make these
datasets extremely valuable to the community at large.

This project was limited by the [relatively few|small number of]
experiments performed [for each type of cardiac stressing | so far]. This
study was also limited in [|direct] clinical translation because of the
animal torso shape and other anatomical features. Future directions of this
project will include [expanding the number of | more] experiments performed
with [each different type of cardiac stressing | both stress protocols].


\balance


\section*{Acknowledgements}  
% This section is not numbered.
% 
Support for this research comes from the NIH NIGMS Center for Integrative
Biomedical Computing (www.sci.utah.edu/cibc), NIH NIGMS grant
no. P41GM103545 and the Nora Eccles Treadwell Foundation for Cardiovascular
Research.


% LateX generates the ``References'' heading automatically and switches
% to 9 point type for the bibliography.  If you use BibTeX (recommended),
% follow the examples in the sample 'refs.bib' file to enter your references,
% and leave the following line unchanged.
\bibliography{library.bib}
\bibliographystyle{cinc}

% If you don't use BibTeX, comment out or remove the previous line, and
% uncomment the following line and the ``}\end{bibliography}'' line below:

% LaTeX inserts the ``Address for correspondence'' heading.
\begin{correspondence}
Brian Zenger\\
72 Central Campus Dr, Salt Lake City, UT 84112\\
zenger@sci.utah.edu
\end{correspondence}

%\end{document}


% % Remove the '%' from the previous line before formatting the final
% % version of your paper
% 
% \clearpage
% \setcounter{section}{-1}
% 
% \section{Instructions for preparing your paper}
% 
% \subsection{This document}
% 
% Please save this document and use the outline on the first page to
% prepare your paper.
% 
% Before formatting your paper, remove these
% instructions by uncommenting the \verb+\end{document}+ line that you
% will find in this document a few lines above this paragraph.
% 
% \subsection{Paper length}
% 
% Please limit your paper to no more than four pages.
% 
% \subsection{LaTeX formatting}
% 
% In order to use this template to create a paper in proper CinC
% style, you will also need to have the CinC LaTeX macros, which
% can be downloaded from http://www.cinc.org/authors\_kit/ in either
% .tar.gz or .zip format.  These files include instructions on
% how to use them, together with complete example papers
% illustrating how to include equations, figures, tables, and
% references.
% 
% \subsection{Title}
% 
% Avoid abbreviations and keep to one or two lines. Remember that the
% title should be easily understood when cited as a reference in another
% publication.
% 
% \balance % equalize column lengths -- see comments on final page below
% 
% \subsection{Section numbering}
% 
% Number your sections as illustrated, \emph{starting with 1.}
% % LaTeX does this automatically.
% 
% 
% \subsection{Tables and figures}
% 
% Tables and figures can fit across both columns if necessary.  Captions go
% above tables, but beneath figures.
% 
% 
% 
% 
% \subsection{Citations and references}
% 
% All references should be included in the text in square brackets in
% their order of appearance, e.g. [1] [1,2] [1--4]. In the reference list
% use the Vancouver style (see BMJ 1991;302:338--41 or New Engl J Med
% 1991;324:424--8).
% 
% References to Computers in Cardiology proceedings should now include
% the volume number. For volumes before 1997, set out as for book
% chapters giving publisher and place of publication, noting that the
% publisher changed from IEEE Computer Society Press to IEEE in 1995.
% 
% There are (at least) two ways to prepare references.  We recommend
% that you use BibTeX if you can, since it automatically numbers your
% citations and generates a properly sorted reference list in
% CinC format, complete with a section heading.
% 
% If you do not use BibTeX, prepare your list of references following
% the commented-out examples on the first page of this document.
% 
% 
% \subsection{Final page}
% 
% The text on the final page should be arranged so that both columns
% are approximately the same height.   Insert \verb+\balance+ anywhere
% within the first column, and LaTeX will
% even out the columns automatically.
% 
% 
% \subsection{End of instructions}
% 
% \emph{All of section 0 should be deleted before you format the
% final version of your paper.}
% 
\end{document}
