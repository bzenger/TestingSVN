% -*-latex-*-
% Document name: significance.tex
% Creator: Rob MacLeod [macleod@cvrti.utah.edu]
%%%%%%%%%%%%%%%%%%%%%%%%%%%%%%%%%%%%%%%%%%%%%%%%%%%%%%%%%%%%%%%%%%%%%%
% Explain the importance of the problem or critical barrier to progress in
% the field that the proposed project addresses.  Explain how the proposed
% project will improve scientific knowledge, technical capability, and/or
% clinical practice in one or more broad fields.  Describe how the
% concepts, methods, technologies, treatments, services, or preventative
% interventions that drive this field will be changed if the proposed aims
% are achieved.
%%%%%%%%%%%%%%%%%%%%%%%%%%%%%%%%%%%%%%%%%%%%%%%%%%%%%%%%%%%%%%%%%%%%%%
\subsection{Significance}
\label{sec:signif}

The most common reason for a visit to the emergency department is chest
pain caused by myocardial ischemia\cite{BLZ:Saf2018,BLZ:Bhu2010},
which occurs when the perfusion to a specific region of the
heart is inadequate.\cite{BMB:Hea94,BMB:Fal2007} Such inadequate perfusion,
in combination with cardiac stress, leads to acute changes in the tissue
that eventually lead to cell death. Well before this cell death,
an insufficient supply of oxygen and other nutrients in combination with poor
metabolite removal creates a toxic extracellular milieu that prevents
cardiomyocytes from functioning normally.\cite{BMB:Kat2011,BMB:Foz86} These
acute transient ischemic events cause the stereotypical crushing chest pain
that ceases after rest, \ie{} after removal of the cardiac
stress.\cite{BMB:Kat2011,BMB:Sur2011b}

A surprising range of pathologies can cause acute transient ischemia,
including coronary artery disease, coronary microvascular dysfunction,
Takotsubo cardiomyopathy, and coronary artery dissection.\cite{BLZ:Saf2018,BLZ:Jes2013,BLZ:Noe2017,BLZ:Jes2012} Each
pathology carries a significant risk of short- and long-term mortality that
can be reduced substantially by early detection.\cite{BLZ:Noe2017}
\textbf{Therefore, detecting myocardial ischemia early is paramount to prevent
  long-term negative consequences.}
\cite{BMB:Kon99,BLZ:Saf2018,BLZ:Knu2018a}

Detecting acute ischemia requires cardiac stress, which is often defined
as inadequate perfusion for the cardiac metabolic demand of a region of the
heart.\cite{BLZ:Pue2004} In most clinical circumstances, the safest method
to induce and detect myocardial ischemia is to increase metabolic demand by
having the patient exercise or by administering a pharmacological infusion
to increase heart rate and
contractility.\cite{RSM:Ste2002,BLZ:Saf2018,BLZ:Knu2018a}

Once the heart has been stressed, transient myocardial ischemia can be
detected by several different markers. Initial elevations of extracellular
potassium concentrations and anaerobic metabolism alter electrical action
potentials and are followed by changes to mechanical contraction, leading to
regional hypokenesis and myocardial wall
thinning.\cite{BLZ:Mai1994,RSM:Rei86b} Electrical responses include a
depolarization of the resting membrane potential of the cell, a decrease of
the upstroke velocity of the cardiac action potential, shortening of the
plateau-phase and overall duration of the action potential, and reduction
in its amplitude. \cite{BLZ:Yan1996,BMB:Kat2011} The resulting spatial
gradients of extracellular potentials between healthy and ischemic tissues
cause intracellular current to flow between healthy and ischemic tissue,
known as ``injury currents'', which can be detected from the body
surface.\cite{KKA:Sha97a,KKA:Sha97b,RSM:Bra76,RSM:Bra88} These currents are
largest during the ST segment of an electrocardiogram (ECG) and cause a
positive or negative deflection, depending on the amount and location of
the ischemic tissue and the location of the ECG electrodes.
\cite{BLZ:Jan1980,BLZ:Jan1981} These electrical changes are transient and
manifest only when the tissue is ischemic. This dynamic complex process of
changes rapidly through an ischemic episode and creates three-dimensional
zones of affected tissue with varying shapes and sizes.  Detection and
localization of these ischemic zones make natural targets for diagnosis and
monitoring as they have electrical consequences that can, in principle, be
measured noninvasively. 

Several methods are used clinically to identify these ischemic
zones. Echocardiography uses ultrasound imaging acquired during
pharmacological stress to identify regions of mechanical dysfunction and
wall thinning.\cite{BLZ:Man1988,BLZ:Gre1997,KKA:Gel97} The ultrasound
imaging provides coarse snapshots of the ischemic region, but it is limited to
discrete intervals in time and by its spatial resolution.\cite{BLZ:Saf2018}
The sensitivity and specificity of stress echocardiography are 85\% and
80\%, respectively.\cite{BLZ:Knu2018a} Myocardial perfusion
imaging (nuclear imaging) can also be used to detect ischemia by injecting
a radioisotope during peak cardiac stress (pharmacological or
exercise). Areas of decreased perfusion, which are likely ischemic,
acquire a lower concentration of radioisotopes than regions with
healthy perfusion. Myocardial perfusion imaging has a sensitivity and specificity of
87\% and 70\%, respectively, requires several professionals, entails a high
radiation dose, and is expensive.\cite{BLZ:Knu2018a} Biochemical markers
such as troponin have potential value for measuring ischemia, but they lack
adequate validation and provide almost no temporal or spatial information
about the ischemia induced.\cite{BLZ:Saf2018} The 12-lead electrocardiogram
is perhaps the most commonly used marker during exercise heart stress and
follows well the temporal changes of ischemia over time, but it provides poor
spatial resolution. The sensitivity and specificity of the
electrocardiographic stress test are in the range of 55--65\%.\cite{BLZ:Knu2018a,
  RSM:Ste2002} \textbf{We propose novel approaches that can provide both
  high temporal and spatial resolution and represent a functional imaging
  modality based on high-resolution ECG.}

The first step to improving detection of acute ischemia is to understand
the electrical ischemic sources. Experimental models used previously have
documented the cellular, tissue, and whole organ manifestations of
myocardial ischemia.\cite{BMB:Hol77a,BMB:Hol77b,BLZ:Kle1978,
  RSM:Jan80,KKA:Cin80,RSM:Cha89,RSM:Mac95e,RSM:Mac97} Holland and Brooks
\etal~performed the first detailed and comprehensive animal studies in the
late 1970s and documented the electrical changes that occurred during
myocardial ischemia.\cite{BMB:Hol77a,BMB:Hol77b}, driving much of the
conventional clinical thinking, including the idea that ischemia first
develops near the endocardial border and progresses radially toward the
epicardium.\cite{BMB:Hol77a,BMB:Hol77b,BMB:Kat2011} Recently, our group has
improved this experimental model dramatically by increasing the temporal
sampling and recording electrode density in the myocardium and on the heart
surface.\cite{RSM:Sho2007,RSM:Ara2009,RSM:Ara2011, BMB:Ara2015,BMB:Ara2016}
Our studies have shown sharply different patterns of ischemia that initiate
throughout the myocardial wall, which has motivated a complete reevaluation
of the electrical sources created during ischemia.

Limits to the translational potential of these findings have included the
lack of simultaneous body-surface ECGs and the absence of clinically
relevant ischemia induction. In addition, no studies to our knowledge have
directly compared the ischemic electrical profile using different heart
stressing mechanisms. The most common comparison reported is the use of a
12-lead ECG and echocardiography to create a binary selection of either the
presence or the absence of
ischemia.\cite{BLZ:Sha1998a,BLZ:Dho2000,BLZ:Mai1994} Some researchers have
examined the ischemic zones detected with nuclear medicine techniques, but
they did not measure the electrical presence of ischemia within the
heart.\cite{BLZ:San1998} \textbf{We propose to address these limitations in
  aims 1 and 2, the result of which will be the most detailed and
  translational model of acute ischemia to date. We will measure realistic
  ischemic sources simultaneously within the heart, on the heart surface,
  and on the body surface, and will also determine experimentally the role
  of different cardiac stress mechanisms in the development of ischemia.}

A better understanding of ischemic sources from these experiments will also
help drive a better method of detection. The transient, complex,
three-dimensional nature of ischemia makes it especially difficult to
identify noninvasively. The 12-lead ECG provides a snapshot of the ischemic
zone, if it happens to exist in a location that is monitored by one of the
leads. An imaging-like approach is an attractive and potentially powerful
option to capture the dynamic and complex nature of ischemic sources.

This imaging is known as ECG Imaging
(ECGI)\cite{RSM:Bro97b,RSM:Pul2010,RSM:Rud2015}; analogous to all imaging
modalities, it uses individual electrodes to capture single views of the
electrical potential in the heart and reconstructs them to capture cardiac
electrical activity from multiple directions.  This process is much like a
series of X-rays taken from different angles and then reconstructed to
create a three-dimensional CT scan. ECGI is an approach with several
decades of research that has recently been used in a various clinical
applications, including premature ventricular contraction localization,
atrial fibrillation detection, and other electrical
abnormalities.\cite{RSM:Sha2015,BLZ:Pot2014,BLZ:Dub2015,BLZ:Wan2016a,RSM:Sha2015,BLZ:Wan2018,RSM:Ost97,RSM:Mac98,RSM:Bro99,RSM:Clu2018,RSM:Clu2015}
This approach can provide a noninvasive, continuous, and patient-specific view of the
electrical activity in the heart and thus a functional status that
could dramatically improve diagnostic accuracy and monitoring capability.
As ECGI specialists, we have designed open-source toolkits that have been
applied to a range of pathologies, including acute
ischemia.\cite{BMB:Mac95,BMB:Bur2011,BMB:Bur2018a,BMB:Bur2018b,BLZ:Tat2018}
To provide enough information for an imaging approach, more electrodes in
more locations are required than the standard 12-lead ECG. This approach
has been known for several decades as ``body-surface potential mapping'' or
BSPM, and it provides a complete, but remote, sampling of the myocardial
electrical potentials. \cite{BLZ:Mil1980,BLZ:Fox1979,BLZ:Hor2001,RSM:Koz95}

Despite the potential utility and the clinical need, ECGI has rarely
been applied or validated in the setting of acute transient myocardial
ischemia.  Other groups have used ECGI to study ischemia and conducted
validation studies using clinical imaging. Nielsen \etal~acquired SPECT
cardiac stress testing images and body-surface potentials from four
patients undergoing a standard exercise stress test.\cite{BMB:Nie2013} They
used used ECGI to reconstruct the ischemic zones at the limited resolution
of the 17-segment model of the ventricles and further classified each zone
as endocardial, epicardial, or transmural, findings they confirmed with
cardiac SPECT. The limitations of this study included the absence of
electrical confirmation and the very low time resolution inherent in SPECT.
Similarly, the relatively low resolution of their approach (centimeters
rather than millimeters) hindered the analysis of the complex ischemic
zones that arise and deform rapidly during cardiac stress.

Members of our group have applied ECGI to ischemia \cite{RSM:Mac95} and
have recently created a fundamentally novel mathematical
formulation based on a new source model to localize the ischemic source
within the myo\-card\-ium.\cite{RSM:Wan2013,RSM:Wan2011a} This advance also
required novel use of numerical solvers that created a very promising
framework to detect acute myocardial ischemia at the level of the
transmembrane potential. \textbf{These methods showed promise but were
  limited by the lack of ground truth datasets from experiments of suitable
  quality. With our expertise and recent improvements to the experimental
  model, we are perfectly poised to implement novel detection techniques to
  identify ischemic sources in the heart. Accomplishing aim 3 would be
  significant because it would result in a new comprehensive and accurate
  method to reconstruct the spatiotemporal image of acute myocardial
  ischemia.}

In summary, this project is significant for several reasons. First, the
novel experimental preparation will be a first of its kind, with high-resolution recordings and transient ischemic control that maximize the
potential to translate any findings into clinical practice. Second, we
will compare and contrast different, clinically relevant types of ischemic
stress in a highly instrumented and physiologically relevant animal
preparation to better detect differences that could never before be
seen. Finally, we will apply novel ECGI techniques to improve the detection
and localization of ischemia, which will better inform physicians of the
progression and status of acute ischemia in a patient-specific manner and
thus improve the unacceptably poor performance of current clinical methods.

