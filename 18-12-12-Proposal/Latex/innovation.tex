% -*-latex-*-
% Filename: innovation.tex
% Last update: Sat Jan 16 14:13:00 2010 by Rob Macleod
%    - created
%
%%%%%%%%%%%%%%%%%%%%%%%%%%%%%%%%%%%%%%%%%%%%%%%%%%%%%%%%%%%%%%%%%%%%%%
%    Does the application challenge and seek to shift current research or
%    clinical practice paradigms by utilizing novel theoretical concepts,
%    approaches or methodologies, instrumentation, or interventions? Are
%    the concepts, approaches or methodologies, instrumentation, or
%    interventions novel to one field of research or novel in a broad
%    sense? Is a refinement, improvement, or new application of theoretical
%    concepts, approaches or methodologies, instrumentation, or
%    interventions proposed?
%%%%%%%%%%%%%%%%%%%%%%%%%%%%%%%%%%%%%%%%%%%%%%%%%%%%%%%%%%%%%%%%%%%%%%
\section{Innovation}
\label{sec:innov}

The goal of this project is to increase the
translational potential of our novel experimental model by recording
simultaneously within the heart, on the epicardial surface, and on the
body surface in high-resolution, while maintaining ischemic control. This
is a novel achievement not previously reported. We will also
provide innovation in physiology by challenging the clinical
assumption that various cardiac stress mechanisms create identical regions
of ischemia. We will be the first to test this clinical assumption in
the same subjects using a physiologically translatable,
large mammal experimental model with high-resolution electrical recordings and the two most common clinical
stressing methods, exercise and pharmacological infusion. Finally, we will
implement and validate a novel ECGI method to noninvasively detect and
localize ischemic sources from the body surface. These innovations will
represent advances in methods, physiology, and clinical translation and
will contribute to a better understanding and detection of a very common and impactful disease state.
