% -*-latex-*-
% Document name: aims.tex
% Creator: Brian Zenger
%
%  %%%%%%%%%%%%%%%%%%%%%%%%%%%%%%%%%%%%%%%%%%%%%%%%%%%%%%%%%%%%%%%%%%%%%

%\documentclass[11pt]{report}
% Read the nih-proposal.sty file to see what this package does to set up
% the document.
%\usepackage{nih-proposal}

%\begin{document}

\section{Specific Aims}

The \textit{most} common reason for a patient to visit the emergency
department is chest pain caused by myocardial ischemia. Myocardial ischemia
develops from inadequate perfusion of myocardial tissue and indicates the
presence of coronary artery disease, ischemic heart disease, or other
potentially fatal cardiac diseases. Current noninvasive tests to detect
ischemia may be limited by inco

The long-term g
      ischemia under different forms of cardiac stress.
	
    \item [Aim 3:] \textbf{Test the hypothesis that ECG imaging techniques
E160024

%\end{itemize}
%\end{numlist}


Success in this project will lead to improvements
in detection and clinical diagnosis of myocardial ischemia using ECGI
techniques validated with experimental data. Our experimental
approach will be valuable for various cardiac modeling applications,
including validation studies for other ECGI techniques. By comparing
exercise and pharmacological stress at the fine scale, our experiments will be the first high-resolution analysis and will increase our understanding of myocardial ischemia development
to guide the
interpretation of clinical stress tests. Finally, as we detect ischemia
early and accurately, we will limit patient burden and increase therapeutic
success.  Conducting these studies in this rich
mentoring environment will ensure deep exposure to all aspects of research
and facilitate my development as an independent physician scientist.

%Paragraph 4: Success in this endeavor will have immediate and profound
%effects on the millions of patients who suffer from this disease.
%Moreover, our hypothesis suggest a novel mechanism for the development of
%the disease that will open new opportunities for therapeutic interventions.



%\end{document}

%Despite the increased understanding of electrical indicators within myocardial ischemia, where myocardial tissue supply of nutrients is less than the amount needed for normal heart function, little is understood about how to clinically identify these parameters. Current tests used to detect these electrical changes are induced via exercise stressing the heart and recording electrical changes. Yet the most sensitive tests for myocardial ischemia, pharmacological stress testing with combined ultrasound imaging, have shown little to no change in common ischemia defining electrical indicators. With the confirmed identification of ischemia, but no indication via electrical indicators, healthcare professionals largely disregard potentially key information about heart status and dismiss their possible value as a diagnostic tool. To continue our advancement in understanding of ischemia, and improve diagnostic testing and identification, this disparity of detection between different tests must be addressed. 

%This limited understanding of the electrical signals that indicate myocardial ischemia hinder our ability to adequately diagnose and treat patients suffering 

%Despite recent advances in cardiac treatment, undetected myocardial ischemia remains a high cause of mortality in the United States. Ischemia is a condition where the myocardial tissue's demand for nutrients is larger than the available supply. Myocardial ischemia plays a role in decreasing heart function, development of heart failure, and inducing fatal cardiac arrhythmias. Knowing these consequences, physicians and health care professionals attempt to detect ischemia using several different methods such as exercise stress testing, nuclear imaging, or pharmacological stress testing. Despite the multitude of testing, ischemia While each test is used to detect ischemia, electrical indicators of myocardial ischemia vary greatly depending on the test. Although the clinical community recognizes this variation, little is understood about why it exists. Insights into these differences may provide new information in detecting and diagnosing myocardial ischemia to help mitigate and prevent long term damage. 


%Pharmacological stress tests use drugs, typically dobutamine, to induce cardiac stress by increasing heart rate and cardiac muscle contraction force. Myocardial ischemia is detected during these pharmacological stress tests by ultrasound imaging or echocardiography alone neglecting a key cardiac function indicator, heart electrical cardiac signals. Most analysis of myocardial electrical signals have been dismissed for perceived decreased sensitivity compared to imaging. However, detection of myocardial ischemia via pharmacological stress testing sensitivity and specificity decent at best with many cases of ischemia not detected. In addition, echocardiography is a complicated task requiring several experts and increased resources to detect ischemic indicators.

%To bridge this gap in understanding indicators of myocardial ischemia we propose a study to investigate the electrical signals of the heart during routine pharmacological cardiac stress testing compared to routine exercise stress testing. Our underlying hypothesis is that measured heart electrical signals change significantly and reproducibly under pharmacological stress testing during myocardial ischemia much like during exercise induced ischemia. To test this hypothesis we will develop an acute myocardial ischemia animal model with high density electrical recording equipment on the surface of the heart and within the myocardium to detect subtle changes in the electrical signals during myocardial ischemia. Next, we will analyze these signals and determine indicators for ischemia as detected by the electrogram recordings. Finally, we will measure body surface electrical signals and identify indicators of ischemia, during a pharmacological stress test, on the body surface. 
