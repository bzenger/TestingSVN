% -*-latex-*-
% Document name: aims.tex
% Creator: Brian Zenger
%
%  %%%%%%%%%%%%%%%%%%%%%%%%%%%%%%%%%%%%%%%%%%%%%%%%%%%%%%%%%%%%%%%%%%%%%

%\documentclass[11pt]{report}
% Read the nih-proposal.sty file to see what this package does to set up
% the document.
%\usepackage{nih-proposal}

%\begin{document}

\section{Specific Aims}

The \textit{most} common reason for a patient to visit the emergency
department is chest pain caused by myocardial ischemia. Myocardial ischemia
develops from inadequate perfusion of myocardial tissue and indicates the
presence of coronary artery disease, ischemic heart disease, or other
potentially fatal cardiac diseases. Current noninvasive tests to detect

%Pharmacological stress tests use drugs, typically dobutamine, to induce cardiac stress by increasing heart rate and cardiac muscle contraction force. Myocardial ischemia is detected during these pharmacological stress tests by ultrasound imaging or echocardiography alone neglecting a key cardiac function indicator, heart electrical cardiac signals. Most analysis of myocardial electrical signals have been dismissed for perceived decreased sensitivity compared to imaging. However, detection of myocardial ischemia via pharmacological stress testing sensitivity and specificity decent at best with many cases of ischemia not detected. In addition, echocardiography is a complicated task requiring several experts and increased resources to detect ischemic indicators.

%To bridge this gap in understanding indicators of myocardial ischemia we propose a study to investigate the electrical signals of the heart during routine pharmacological cardiac stress testing compared to routine exercise stress testing. Our underlying hypothesis is that measured heart electrical signals change significantly and reproducibly under pharmacological stress testing during myocardial ischemia much like during exercise induced ischemia. To test this hypothesis we will develop an acute myocardial ischemia animal model with high density electrical recording equipment on the surface of the heart and within the myocardium to detect subtle changes in the electrical signals during myocardial ischemia. Next, we will analyze these signals and determine indicators for ischemia as detected by the electrogram recordings. Finally, we will measure body surface electrical signals and identify indicators of ischemia, during a pharmacological stress test, on the body surface. 
