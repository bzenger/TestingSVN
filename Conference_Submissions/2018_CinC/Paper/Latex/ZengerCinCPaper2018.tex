% -*- Mode:TeX -*-
% LaTeX template for CinC papers                   v 1.1a 22 August 2010
%
% To use this template successfully, you must have downloaded and unpacked:
%       http://www.cinc.org/authors_kit/papers/latex.tar.gz
% or the same package in zip format:
%       http://www.cinc.org/authors_kit/papers/latex.zip
% See the README included in this package for instructions.
%
% If you have questions, comments or suggestions about this file, please
% send me a note!  George Moody (george@mit.edu)
%
\documentclass[twocolumn]{cinc}
\usepackage{graphicx}
\newcommand{\mapthreed}{\textit{map3d }}
\begin{document}

% Keep the title short enough to fit on a single line if possible.
% Don't end it with a full stop (period).  Don't use ALL CAPS.
\title{Electrocardiographic Comparison of Dobutamine and Bruce Cardiac \\ Stress Testing with High Resolution Mapping in Experimental Models}

% Both authors and affiliations go in the \author{ ... } block.
% List initials and surnames of authors, no full stops (periods),
%  titles, or degrees.
% Don't use ALL CAPS, and don't use ``and'' before the name of the
%  last author.
% Leave an empty line between authors and affiliations.
% List affiliations, city, [state or province,] country only
%  (no street addresses or postcodes).
% If there are multiple affiliations, use superscript numerals to associate
%  each author with his or her affiliations, as in the example below.

\author { Brian Zenger$^{1,2,3,4}$, Wilson W. Good$^{1,2,3}$, Rob S. MacLeod$^{1,2,3}$\\
\ \\ % leave an empty line between authors and affiliation
$^1$ Scientific Computing and Imaging Institute, University of Utah, SLC, UT, USA \\
$^2$  Nora Eccles Cardiovascular Research and Training Institute, University of Utah, SLC, UT, USA \\
$^3$ Department of Bioengineering, University of Utah, SLC, UT, USA \\
$^4$ School of Medicine, University of Utah, SLC, UT, USA }

\maketitle

% LaTeX inserts the ``Abstract'' heading in the proper style and
% sets the text of the abstract in italics as required.
\begin{abstract}

 Clinical tests to detect acute myocardial ischemia induce transient cardiac stress by means of exercise or pharmaceutical stimulation and measure electrical changes of the heart on the body surface via an electrocardiogram (ECG).  Such tests  assume that both mechanisms induce identical —or at least similar— forms of ischemia. To improve ECG detection of myocardial ischemia, we must study how varied stressing agents (pharmacological or paced stressors) change ECG signatures. Electrical recordings within the myocardium, on the epicardial surface, and on the body surface of a porcine model of acute, controlled ischemia were measured during ischemic episodes. These  episodes were induced with constant hydraulic occlusion of the left anterior descending coronary artery at 10\% of normal flow rates. Heart stress was modulated either via pacing with clinical standard BRUCE test protocol heart rates, or by clinical standard continuous dobutamine infusions. Each episode lasted  15 minutes with stepwise increase in pacing rate or pharmacological infusion rate every 3 minutes. Preliminary qualitative results suggest significant differences in the recorded electrical signal between pacing and pharmacological stress mechanisms. Differences include the location and volume of ischemia and its  temporal progression measured the heart. These results, although preliminary, show that significant differences occur in the electrical signatures of myocardial ischemia depending on the type of stress placed on the heart. 




\end{abstract}
% LaTeX inserts the extra space here automatically.

\section{Introduction}
% Section numbering is automatic.  The examples on the next page
% illustrate how to make subsections.

Ischemic heart disease is one of the most common heart pathologies, effecting over 8 million people globally. \cite{Roth2015} Myocardial ischemia occurs when the demand for nutrients and perfusion by the heart outweighs the available supply. This creates a supply-demand mismatch that can lead to devastating long term consequences including increased risk for myocardial infarction, cardiac arrhythmia, and sudden cardiac death.\cite{Roth2015} For decades the electrocardiogram (ECG) has been the primary acute detection method for myocardial ischemia. \cite{McCarthy1990} However, current ECG methods used to detect myocardial ischemia are mediocre at best, with reported sensitivity and specificity ranging from 50-72\% and 69-90\%, respectively. \cite{Akkerhuis2011} This poor ECG performance to detect myocardial ischemia indicates that many patients are released from clinical care unaware of their potentially life-threatening condition. Improvements in the electrical detection of myocardial ischemia must be made to ensure patients and physicians can be confident in diagnosing and treating myocardial ischemia early to prevent potentially fatal long term consequences. 

.    


One possible source of the poor performance of ECG to detect and isolate myocardial ischemia from ECG originate from different cardiac stressing mechanisms. Clinical tests to detect acute myocardial ischemia induce transient cardiac stress by means of exercise or pharmaceutical stimulation and measure electrical changes of the heart on the body surface via an  ECG. Such tests  assume that both mechanisms induce identical—or at least similar—forms of ischemia. However, no definitive experiments have been performed to assess the electrical differences produced during different stressing mechanisms. For example, continuous dobutamine infusion rarely produce the classical ECG signatures usually associated with myocardial ischemia but during an exercise stress test, the classical ECG signatures are present. These contradictions indicate that the methods of inducing ischemia may not produce identical regions of ischemia and substantiate controlled examination of the methods used to stress the heart. 

To improve ECG detection of myocardial ischemia, we must study how varied stressing agents (pharmacological or paced stressors) change ECG signatures. For this project, our goal was to test the differences in myocardial ischemia development under controlled experimental conditions. We tested two cardiac stressing mechanisms, pacing the heart following average heart rates of a BRUCE protocol and continuous dobutamine infusion rates of a standard clinical protocol. 


%
%\begin{figure}[h]
%%\centering
%\includegraphics[width=7.9cm]{graph.png}
%\caption{Put the figure legend here, clearly describing the figure.}
%\label{FIGURA1}
%\end{figure}

%Always leave a line space after a figure legend. Avoid background colors as they can make printed figures hard to read.

\section{Methods}

\subsection{Animal Model} 

Swine and canine animal models were selected for this experimental preparation. Both species were chosen because of their similar cardiac anatomy, electrical system, and vascular structure to humans. The animals of each species were 25-35 kg in weight and 8 months to several years of age. The animals were purpose bread for the use in experimental research. All studies were approved by the Institutional Animal Care and Use Committee at the University of Utah and conformed to the Guide for Care and Use of Laboratory Animals. After 12 hours of fasting the animals were sedated using an intravenous propofol bolus of 5-8 mg/Kg in canines or a mixture of Telazol (4.4 mg/kg), Ketamine (2.2 mg/kg), and Xylazine (2.2 mg/kg) in swine through intravenous access and then intubated. Once intubated, isoflurane gas (1-5\%) was used for anesthesia. At the end of the experiment animals were euthanized while under general anesthesia, with intravenous Beuthanasia 1 ml/10 Kg. The heart was then removed for further evaluation.

\subsection{Surgical Procedure}

Following sedation, a sternotomy was performed to expose the thoracic cavity. The pericardium was opened and the heart was suspended in a pericardial cradle. Following exposure, a portion of the left anterior descending (LAD) coronary artery was dissected and a calibrated hydraulic occluder (Access Technologies, Skokie, IL, USA) was placed around the dissected portion of the LAD coronary. An atrial pacing clip was then placed on the appendage of the right atria. Following placement of the electrical recording equipment (described below), the pericardium was sutured closed and the sternum was wired and sutured together. To limit air within the volume conductor, chest tubes were tunneled into the mediastinal, pleural, and pericardial cavities and held under constant vacuum suction. The outer layers of dermis were sutured closed and checked for potential separations. Standard laboratory values were measured and recorded throughout the experiment including blood pH, PaCO2, oxygen saturation, temperature, and blood pressure.  

\subsection{Electrical Recording Equipment}

\subsubsection{Electrode Arrays}

Electrical recording equipment was all custom build at the Nora Eccles Treadwell Cardiovascular Research and Training Institute (CVRTI). The electrical signals within the myocardium were measured using transmural plunge needle arrays 21.2 mm in length for the left ventricle and 15 mm in length for the right ventricle with 10 electrodes evenly spaced down the shaft of each needle. The spacing of electrodes in the left ventricular plunge needles was 1.6 mm and the spacing of the needles in the right ventricle was 1.0 mm.  For these experiments 12-25 needles were placed in the assumed perfusion bed of the LAD coronary and concentrated on the anterior surface of the heart. The epicardial potentials were measured using a 247 electrode sock array. This array is created with evenly spaced electrodes stitched into a nylon stocking material. The distance between sock electrodes is approximately 10 mm. The torso surface electrodes are in linear strips of 12 electrodes evenly spaced at 3 cm apart. Each electrode had an 11 mm diameter Ag-AgCl sensor embedded in an epoxy housing with a 2 mm deep gel cavity. The number of strips applied to the torso surface varied between 6-10 electrode strips (72-120 electrodes) depending on surface area available for each animal. 

\subsubsection{Data Acquisition}

The potentials from the sock, needle, and torso surface electrodes were recorded using a custom acquisition system. This system could record simultaneously from 1024 channels at 1 kHz sampling rate and 12 bit resolution. Briefly, the acquisition system consisted of multiplexers, interface circuitry, and a personal computer (PC) hosting a custom software written in Labview (National Instruments, Austin, TX, USA) that managed the hardware and allowed continuous signal acquisition. A band pass filter with cutoff frequencies at 0.03 and 500 Hz avoided both DC potentials and aliasing. Wilson's central terminal leads were used as the remote reference for all the unipolar signals recorded from the sock needle, and torso surface electrodes. Electrical recordings were continuously recorded from all 1024 channels for the duration of the experiment. Prior to each experiment, calibration signals were recorded for each channel to be gain adjusted.  


\subsection{Ischemia Intervention Protocols}

During an experiment several transient ischemic interventions could be induced. Each of these interventions lasted between 8-15 minutes and were followed by a 30 minute rest period. During interventions modifications to heart rate and LAD occlusion could be made to simulate varying levels of ischemic stress. An example ischemic protocol performed modeled the Bruce exercise stress test protocol by increasing heart rate a set amount above resting heart rate every three minutes for fifteen minutes. This increase in heart rate was predetermined from average increased heart rates during Bruce stress protocols reported in the literature. \cite{Okin1986a} The occlusion percentage was fixed throughout the 15 minute time interval. The intervention was terminated with the presence of three or more premature ventricular contractions in series. Another example of an ischemic protocol performed was a standard dobutamine stress test protocol. Dobutamine has been used as a pharmacological stress agent clinically because of its direct stimulation of beta-1 sympathetic receptors on the heart causing increased heart rate and contractility. For this intervention, again the occlusion was fixed and the model was continuously infused with dobutamine at a set dose for 3 minutes. Dosages used followed the standard dobutamine stress testing protocols. \cite{Secknus1997} This intervention again lasted 15 minutes or until a series of three premature ventricular contractions occurred. 

\subsection{Image Acquisition and Processing}

\subsubsection{Image Acquisition}

After each experiment, the heart was excised and scanned with a 7 Tesla MRI scanner (Bruker BIOSPEC 70/30, Billerica, MA) using FISP (Fast Imaging with  Steady-state Precession) and FLASH (Fast Low Angle Shot) imaging  sequences. To visualize fiber orientation the heart, a diffusion-weighted MRI sequence was also performed for each excised heart. To visualize the perfusion bed and vasculature, the coronaries were perfused with BriteVue contrast agent (Scarlet Imaging, SLC, UT, USA) and the heart was scanned using an Inveon microCT Scanner (Siemens, Munich, Germany). 

\subsubsection{Image Segmentation}

Capitalizing on the combined advantages of both FISP (consistent volume boundaries) and FLASH sequences (high internal contrast), we  produced realistic geometric segmentations of cardiac tissue, blood, and  transmural plunge needle geometries using the Seg3D (SCI Institute, SLC, UT, USA) open-source software package. Perfusion bed and coronary structure could also be segmented from high resolution CT imaging using the Seg3D software. 

%\subsubsection{Mesh Generation}
%
%Segmented cardiac volumes served as inputs to the BioMesh3D open-source software package (SCI Institute, SLC, UT, USA) or the Cleaver open-source software package (SCI Institute, SLC, UT, USA) which we employed to generate realistic, unstructured, three-dimensional, tetrahedral meshes for use in subsequent finite element simulations. These packages incorporate a particle-based approach that optimizes surface node locations and then applies Delaunay tessellation to optimize element quality. Resulting cardiac meshes contained 3 to 4 million elements with smooth conforming interfaces along material boundaries.

\subsection{Geometric Registration}

%\subsubsection{Landmark Point Recordings}

At the conclusion of each experiment, the locations of the linear torso surface electrode strips, preselected sock electrodes, transmural plunge needles on the cardiac surface were digitally recorded using a Microscribe three-dimensional digitizer (Solution Technologies, Oella, MD, USA). In addition, landmark sites including the location of the occlusion site, distribution of major epicardial coronary arteries, and the outline of the myocardial shape were also captured using the digitizer. Once the locations of the plunge needles were recorded, they were replaced with plastic spacers. These spacers prevented closure of the tissue holes created by the needles, and were visualized in the acquired anatomical MRI scans, providing information for electrode location.

%\subsubsection{Registration Pipeline}
%
%The recorded landmark points and the acquired MRI scans from the segmented transmural plunge needles were registered together with an affine transformation. Once transformed the sock... . The same transformation was applied to a three-dimensional mesh of the electrode sock array. ??? Need to talk about the information here. ???


\subsection{Signal Processing}

The electrical signals recorded during the study were processed in Preprocessing Framework for Electrograms Intermittently Fiducialized from Experimental Recordings (PFEIFER) an open-source MATLAB-based signal processing platform designed to process bioelectric signals acquired from experiments that include recording electrodes placed in or around the heart or on the body surface. \cite{Rodenhauser2018} Using PFEIFER we were able to calibrate, baseline correct, filter, and mark specific time instances within cardiac signals for analysis. Calibration was performed on each individual channel by acquiring a reference sine wave of set amplitude(s) recorded prior to the experiments. These acquired signals where then used to extract a correction factor to be applied to all the recorded signals, resulting in a gain adjustment. A linear baseline drift correction was applied by selecting two assumed isoelectric points at the beginning and end of an electrical signal. Poor quality signals from sock and needle electrodes were discarded. Signals recorded from needle electrodes that were without initial positive wave deflection were identified as electrodes within the blood pool of the ventricles were also discarded. The global root mean squared (RMS) signal was computed from all the sock, transmural plunge needle, and torso surface electrodes and then used to identify unique markings within the cardiac signal. The points identified were the QRS waveform, T waveform, and peak of the T waveform. From these points features such as ST40\%, T-peak, QRS amplitude, etc., were extracted from the cardiac signals. PEFEIFER also has the capability of applying template fiducials points marked in one portion of the signal to the entire signal. This greatly reduces user input and signal processing time for each experiment. 

\subsection{Electrical Potential Mapping and Data Visualization}

Finally the processed experimental data was mapped to the identified electrode locations within the heart, on the epicardial surface of the heart, and on the torso surface. Full experimental model datasets were then visualized using \mapthreed (SCI Institute, SLC, UT, USA) or SCIRun (SCI Institute, SLC, UT, USA) open-source software packages. This allowed for extensive spacial exploration of ischemia and hypothesis development. 

\section{Results}

Preliminary qualitative results indicate significant changes in the spatial distribution of ischemic potentials within the heart, on the heart surface, and on the body surface. This spatial distribution changes are accurately sampled on the cardiac plunge needles and do not vary. 

Preliminary qualitative results also indicate significant changes in the temporal development of myocardial ischemia. The ischemic potentials develop earlier and at a regular rate during BRUCE paced cardiac stressing compared to dobutamine stress testing. 

\section{Discussion}

Current ECG metrics do not distinguish between different types of cardiac stressing. Despite this obvious need for investigation, clinicians have been applying identical ECG metrics for each variable mode of cardiac stress. This study has shown that individual stress mechanisms produce different regions and development patterns of ischemia. These findings suggest different stresses on the heart require unique diagnostic criteria to detect and monitor myocardial ischemia. This finding underscores a recurring theme that the process of ischemia detection and development is more complicated than conventional explanations have indicated. Previous studies similar to this have shown significant deviations from conventional clinical notions and metrics of myocardial ischemia. 

This project was limited by the relatively few experiments performed for each type of cardiac stressing. This study was also limited in clinical translation because of the animal torso shape and other anatomical features. Future directions of this project will include expanding the number of experiments performed with each different type of cardiac stressing. 


\balance


\section*{Acknowledgements}  
% This section is not numbered.
% 
Support for this research comes from the NIH NIGMS Center for Integrative
Biomedical Computing (www.sci.utah.edu/cibc), NIH NIGMS grant
no. P41GM103545, the Nora Eccles Treadwell Foundation for Cardiovascular Research.


% LateX generates the ``References'' heading automatically and switches
% to 9 point type for the bibliography.  If you use BibTeX (recommended),
% follow the examples in the sample 'refs.bib' file to enter your references,
% and leave the following line unchanged.
\bibliography{library.bib}
\bibliographystyle{cinc}

% If you don't use BibTeX, comment out or remove the previous line, and
% uncomment the following line and the ``}\end{bibliography}'' line below:

% LaTeX inserts the ``Address for correspondence'' heading.
\begin{correspondence}
Brian Zenger\\
72 Central Campus Dr, Salt Lake City, UT 84112\\
zenger@sci.utah.edu
\end{correspondence}

%\end{document}


% % Remove the '%' from the previous line before formatting the final
% % version of your paper
% 
% \clearpage
% \setcounter{section}{-1}
% 
% \section{Instructions for preparing your paper}
% 
% \subsection{This document}
% 
% Please save this document and use the outline on the first page to
% prepare your paper.
% 
% Before formatting your paper, remove these
% instructions by uncommenting the \verb+\end{document}+ line that you
% will find in this document a few lines above this paragraph.
% 
% \subsection{Paper length}
% 
% Please limit your paper to no more than four pages.
% 
% \subsection{LaTeX formatting}
% 
% In order to use this template to create a paper in proper CinC
% style, you will also need to have the CinC LaTeX macros, which
% can be downloaded from http://www.cinc.org/authors\_kit/ in either
% .tar.gz or .zip format.  These files include instructions on
% how to use them, together with complete example papers
% illustrating how to include equations, figures, tables, and
% references.
% 
% \subsection{Title}
% 
% Avoid abbreviations and keep to one or two lines. Remember that the
% title should be easily understood when cited as a reference in another
% publication.
% 
% \balance % equalize column lengths -- see comments on final page below
% 
% \subsection{Section numbering}
% 
% Number your sections as illustrated, \emph{starting with 1.}
% % LaTeX does this automatically.
% 
% 
% \subsection{Tables and figures}
% 
% Tables and figures can fit across both columns if necessary.  Captions go
% above tables, but beneath figures.
% 
% 
% 
% 
% \subsection{Citations and references}
% 
% All references should be included in the text in square brackets in
% their order of appearance, e.g. [1] [1,2] [1--4]. In the reference list
% use the Vancouver style (see BMJ 1991;302:338--41 or New Engl J Med
% 1991;324:424--8).
% 
% References to Computers in Cardiology proceedings should now include
% the volume number. For volumes before 1997, set out as for book
% chapters giving publisher and place of publication, noting that the
% publisher changed from IEEE Computer Society Press to IEEE in 1995.
% 
% There are (at least) two ways to prepare references.  We recommend
% that you use BibTeX if you can, since it automatically numbers your
% citations and generates a properly sorted reference list in
% CinC format, complete with a section heading.
% 
% If you do not use BibTeX, prepare your list of references following
% the commented-out examples on the first page of this document.
% 
% 
% \subsection{Final page}
% 
% The text on the final page should be arranged so that both columns
% are approximately the same height.   Insert \verb+\balance+ anywhere
% within the first column, and LaTeX will
% even out the columns automatically.
% 
% 
% \subsection{End of instructions}
% 
% \emph{All of section 0 should be deleted before you format the
% final version of your paper.}
% 
\end{document}
